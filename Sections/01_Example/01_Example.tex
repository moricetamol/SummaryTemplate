\documentclass[
../../Summary.tex,
]
{subfiles}
    
\externaldocument[ext:]{../../Summary.tex} 
% !!! If you rename the main file of the summary, change the path to the main file here !!! and in the documentclass options

% Set Graphics Path to folder with images
\graphicspath{{../../Pics}}

\begin{document}
\section{Example Section}
\subsection{Predefined Commands}
Not a lot, but still useful. 

\begin{itemize}
    \item \defc{defc}: Definition color and bold
    \item \mathc{mathc}: Math color
    \item \codec{codec}: Code color and ttfont
\end{itemize}

\subsection{Predefined Box Types}
I have defined multiple tcolorbox environments, to make it easier to use them, without always having to add the options.

In general there are three types of boxes:
\begin{itemize}
    \item \defc{defbox} for definitions, concepts etc.
    \item \mathc{mathbox} for math
    \item \codec{codebox} for code 
\end{itemize}

They mainly differ in color, which can be easily changed in \texttt{summary\_individuals.tex}


For all three types of boxes there are some small different versions:

\subsubsection{Definition Boxes}
\begin{defbox}
    [defbox]
    Normal definition box, with title, spanning the whole width of the environment. Takes title as the first argument and other options as the second argument.
\end{defbox}

\begin{defbox*}
    \defc{defbox*}: Same as \defc{defbox}, but without a title. Can take one argument, which will be used as additional options.
\end{defbox*}

\begin{smalldefbox}
    [smalldefbox]
    Like \defc{defbox} but with the \texttt{hbox} option. Makes it adapt the width to the content.
\end{smalldefbox}

\begin{smalldefbox*}
    \defc{smalldefbox*}: Is to \defc{smalldefbox} as \defc{defbox*} is to \defc{defbox}.
\end{smalldefbox*}

\begin{csdb*}
    \defc{csdb*}: Like \defc{smalldefbox*} but centered to the environment.
\end{csdb*}

\inldefbox{\defc{inldefbox}: Way to put definitions in-line.}

\subsubsection{Math Boxes}
\begin{mathbox}
    [mathbox]
    Normal math box, with title, spanning the whole width of the environment. Takes title as the first argument and other options as the second argument.
\end{mathbox}

\begin{mathbox*}
    \defc{mathbox*}: Same as \defc{mathbox}, but without a title. Can take one argument, which will be used as additional options.
\end{mathbox*}

\begin{smallmathbox}
    [smallmathbox]
    Like \defc{mathbox} but with the \texttt{hbox} option. Makes it adapt the width to the content.
\end{smallmathbox}

\begin{smallmathbox*}
    \defc{smallmathbox*}: Is to \defc{smallmathbox} as \defc{mathbox*} is to \defc{mathbox}.
\end{smallmathbox*}

\begin{csmb*}
    \defc{csmb*}: Like \defc{smallmathbox*} but centered to the environment.
\end{csmb*}

\inlmathbox{\defc{inlmathbox}: Way to put math in-line.}

\subsubsection{Code Boxes}
\begin{codebox}
    [codebox]
    Normal math box, with title, spanning the whole width of the environment. Takes title as the first argument and other options as the second argument.
\end{codebox}

\begin{codebox*}
    \defc{codebox*}: Same as \defc{codebox}, but without a title. Can take one argument, which will be used as additional options.
\end{codebox*}

\begin{smallcodebox}
    [smallcodebox]
    Like \defc{codebox} but with the \texttt{hbox} option. Makes it adapt the width to the content.
\end{smallcodebox}

\begin{smallcodebox*}
    \defc{smallcodebox*}: Is to \defc{smallcodebox} as \defc{codebox*} is to \defc{codebox}.
\end{smallcodebox*}

\begin{cscb*}
    \defc{cscb*}: Like \defc{smallcodebox*} but centered to the environment.
\end{cscb*}

\inlcodebox{\defc{inlcodebox}: Way to put code snippets in-line.} Defaults to ttfont.

\subsection{Code Snippets}
\subsubsection{Actual Code with listings}
For actual code snippets I use the \texttt{listings} package. There are some graphical options in the preamble.

\begin{codebox}
    [Example Java]
    \begin{lstlisting}[language=Java]
public class Example {
    /**
     * This is a docstring
     */
    public static void main(String[] args) {
        // This is a comment
        System.out.println("Hello World!");
        return;
    }
}\end{lstlisting}
\end{codebox}

\begin{codebox}
    [Example Python]
    \begin{lstlisting}[language=Python]
# This is a comment
def example() -> bool:
    """
    This is a docstring
    """
    print("Hello World!")
    return True
\end{lstlisting}
\end{codebox}

\subsubsection{Pseudocode with algorithm2e}
For pseudocode I use \texttt{algorithm2e}. I've modified the style in the preamble to make it look a bit nicer, especially to show scope better. There are also some additional commands I've defined - For these check the preamble.

\begin{codebox}
    [Example Pseudocode]
    \begin{algorithm}[H]
        \SetKwFunction{ExamplePseudocode}{ExamplePseudocode}

        \Fn{\ExamplePseudocode{args}}{
            \tcp{This is a comment}
            \For{int i = 0; i < 10; i = i + 1}{
                \If{condition}{
                    \KwRet{true}
                }
            }
        }
    \end{algorithm}
\end{codebox}

\end{document}